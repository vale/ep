\documentclass[compress,11pt]{beamer}
%\includeonly{pendel}
\usetheme{Ilmenau}
%\usetheme{fau-4-3}
%\usecolortheme{beaver}
%\beamertemplatenavigationsymbolsempty
\usepackage[ngerman]{babel}
\usepackage{marvosym}
\usepackage{multimedia}
\usepackage[utf8]{inputenc}
\usepackage{amsmath}
\usepackage{amsfonts}
\usepackage{amssymb}
\usepackage{graphicx}
\usepackage{esvect}
%\author{}
\title{EP Gruppe 8}
%\setbeamercovered{transparent}
%\setbeamertemplate{navigation symbols}{}
%\logo{}
%\institute{}
%\date{}
%\subject{}
\usepackage{verbatim}
\begin{document}
\section{Sequentielle Logik}
\subsection{RS-Flip-Flop}
\begin{frame}
\begin{block}{Schaltung}
\includegraphics[scale=0.7]{rs}
\end{block}
\end{frame}
\begin{frame}
\begin{block}{Aufbau und Funktion}
\begin{itemize}
\item Zusammengesetzt aus zwei gekoppelten NAND-Gattern
\item Ausgänge sind (vom Zustand $\overline{S} = 0$ und $\overline{R} = 0$ abgesehen) invertiert
\item Durch Setzen von $\overline{S} = 1$ bzw.$\overline{R} = 1$ können die Ausgänge gesetzt weden
\item Wahrheitstafel des NAND-Gatters bewirkt, dass durch das Setzen der beiden Eingänge auf 1 der letzte Zustand des Flip-Flops ausgegeben wird - Flip-Flop kann Zustände speichern
\end{itemize}
\end{block}
\end{frame}
\begin{frame}
Wahrheitstafel:\\
\begin{tabular}{|c|c||c|c|}
\hline 
$\overline{S}$ & $\overline{R}$ & $Q$ & $\overline{Q}$ \\ 
\hline 
0 & 0 & (1) & (1) \\ 

0 & 1 & 1 & 0 \\ 
 
1 & 0 & 0 & 1 \\ 
 
1 & 1 & $Q_{n-1}$ & $\overline{Q_{n-1}}$ \\ 
\hline 
\end{tabular} 




\end{frame}
\subsection{Taktgesteuerter RS}
\begin{frame}

\begin{block}{Schaltung}
\includegraphics[scale=0.7]{taktrs}
\end{block}

\end{frame}
\begin{frame}
\begin{block}{Funktionsweise}
\begin{itemize}
\item Dem RS-Flip:Flop werden nun noch zwei NAND-Gatter vorrangestellt, die über das Clock-Signal verbunden sind
\item Solange $C = 0$ $\Rightarrow$ fester Anfangszustand, der nicht durch S und R beeinflusst wird, da NAND-Gatter $1$ ausgibt, sobald eines der Signale $0$ ist
\item Schaltung ist "flankengesteuert", also gesetzte $S$ und $R$ werden erst übernommen, wenn $C$ eingeschaltet wird
\end{itemize}
\end{block}
\end{frame}
\begin{frame}
\begin{block}{Verlauf der Schaltvorgänge}
\includegraphics[scale=1.3]{trs}
\end{block}

\end{frame}
\subsection{D-Latch}

\begin{frame}
\begin{block}{Schaltung}
\includegraphics[scale=1]{dlatch1}

\end{block}

\end{frame}
\begin{frame}\begin{tabular}{|c|c||c|}
\hline 
$\overline{D}$ & $C$ & $\overline{S}$ \\ 
\hline 
1 & 0 & 1 \\ 

1 & 1 & 1 \\ 

0 & 0 & 1 \\ 

0 & 1 & 0 \\ 
\hline
\end{tabular} \\
Bis auf den irrelevanten Fall $C = 0$ (es findet keine Veränderung des Zustandes statt) sind $\overline{D}$ und $\overline{S}$ identisch
\end{frame}
\begin{frame}
\begin{block}{Schaltung kann somit vereinfacht werden}
\includegraphics[scale=0.7]{dlatcheinfach}
\end{block}
\end{frame}
\begin{frame}
\begin{block}{Funktionsweise}
\begin{itemize}
\item $R$ zu $S$ negiert, daher existiert kein unbestimmter Zustand
\item Schaltung nicht flanken- sondern zustandsgesteuert: sobald $C = 1$, lassen sich Zustände setzen bzw. überschreiben
\end{itemize}
\end{block}
\end{frame}
\begin{frame}
\begin{block}{Verlauf der Schaltvorgänge}
\includegraphics[scale=1.3]{dlatch}
\end{block}
\end{frame}
\begin{frame}
\begin{block}{Wahrheitstafel}
\begin{tabular}{|c|c|c|}
\hline 
$D$ & $C$ & $Q$ \\ 
\hline 
x & 0 & $Q_{n-1}$ \\ 
\hline 
0 & 1 & 0 \\ 
\hline 
1 & 1 & 1 \\ 
\hline 
\end{tabular} 

\end{block}
\end{frame}
\subsection{Flankengesteuertes RS-Flip-Flop}
\begin{frame}
\begin{block}{Schaltung}
\includegraphics[scale=0.7]{flanke}
\end{block}
\end{frame}
\begin{frame}
\begin{block}{Funktionsweise}
\begin{itemize}
\item Bei $C$ konstant wird der zuletzt gesetzte Wert von $Q$ rückgegeben
\item Sobald steigende Flanke auf $C$, wird $Q$ = $D$ gesetzt
\end{itemize}
\end{block}
\end{frame}
\begin{frame}
\begin{block}{Verlauf der Schaltvorgänge}
\includegraphics[scale=1]{triggerrs}
\end{block}
\end{frame}
\begin{frame}
\begin{block}{Wahrheitstafel}
\begin{tabular}{|c|c|c|}
\hline 
$D$ & $C$ & $Q$ \\ 
\hline 
x & 0 & $Q_{n-1}$ \\ 

n & Steigende Flanke & n \\ 
 
x & 1 & $Q_{n-1}$ \\ 
\hline 
\end{tabular} 
\end{block}
\end{frame}


\section{Zähler}
\begin{frame}
\begin{block}{Schaltung}
\includegraphics[scale=0.7]{zahler}
\end{block}
Es kann so Addition der Flankensignale in Binärform realisiert werden
\end{frame}
\begin{frame}

Durchführung mit 2 * 2 flankengetriggerten D-Flip-Flops in IC-Form

\begin{block}{Funktionsweise}
\begin{itemize}
\item $D$ immer mit $Q$ verbunden, also wechselt $Q$ bei jeder Flanke den Wert
\item Da Flip-Flop flankengetriggert, nur Reaktion auf einfaches Betätigen des Schalters
\item Bei Setzen von $Q$ auf 0 wird $\overline{Q}$ auf 1 gesetzt und die anderen Bits erhalten ein Schaltsignal
\end{itemize}
\end{block}
\end{frame}
\begin{frame}
\begin{block}{Eigenschaften}
\begin{itemize}
\item Speicher = 4 Bit, somit ist maximale darstellbare Zahl 15
\item Wenn Wert des Zählers gleich 15, sind alle $Q$ = 1, bei einer weiteren Flanke werden alle $Q$ wieder auf 0 gesetztc $\Rightarrow$ kein Übertrag möglich
\end{itemize}
\end{block}
\end{frame}

\section{binary coded decimal und 7-­Segment­-Anzeige}
\begin{frame}
Ziel: Darstellung der Binärzahl des Zählers als gewohnte Dezimalzahl, Realisierung über speziellen IC sowie kompatibler Anzeige
\begin{block}{Schaltung}
\includegraphics[scale=0.7]{a5sch}

\end{block}
\end{frame}

\end{document}
