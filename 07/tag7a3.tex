\documentclass[compress,11pt]{beamer}
%\includeonly{pendel}
\usetheme{Ilmenau}
%\usetheme{fau-4-3}
%\usecolortheme{beaver}
%\beamertemplatenavigationsymbolsempty
\usepackage[ngerman]{babel}
\usepackage{marvosym}
\usepackage{multimedia}
\usepackage[utf8]{inputenc}
\usepackage{amsmath}
\usepackage{amsfonts}
\usepackage{amssymb}
\usepackage{graphicx}
\usepackage{esvect}
%\author{}
\title{EP Gruppe 8}
%\setbeamercovered{transparent}
%\setbeamertemplate{navigation symbols}{}
%\logo{}
%\institute{}
%\date{}
%\subject{}
\usepackage{verbatim}
\begin{document}
\section{Sequentielle Logik}
\subsection{RS-Flip-Flop}
\begin{frame}
Schaltung:
%\includegraphics[scale=0.7]{•}





\end{frame}
\begin{frame}

\begin{block}{Aufbau und Funktion}
\begin{itemize}
\item Zusammengesetzt aus zwei gekoppelten NAND-Gattern
\item Ausgänge sind (vom Zustand $\overline{S} = 0$ und $\overline{R} = 0$ abgesehen) invertiert
\item Durch Setzen von $\overline{S} = 1$ bzw.$\overline{R} = 1$ können die Ausgänge gesetzt weden
\item Wahrheitstafel des NAND-Gatters bewirkt, dass durch das Setzen der beiden Eingänge auf 1 der letzte Zustand des Flip-Flops ausgegeben wird - Flip-Flop kann Zustände speichern
\end{itemize}
\end{block}
\end{frame}

\begin{frame}
Wahrheitstafel:\\





\end{frame}
\subsection{Taktgesteuerter RS}
\begin{frame}
Schaltung:\\




\end{frame}
\begin{frame}
\begin{block}{Funktionsweise}
\begin{itemize}
\item Dem RS-Flip:Flop werden nun noch zwei NAND-Gatter vorrangestellt, die über das Clock-Signal verbunden sind
\item Solange $C = 0$ $\Rightarrow$ fester Anfangszustand, der nicht durch S und R beeinflusst wird, da NAND-Gatter $1$ ausgibt, sobald eines der Signale $0$ ist

\end{itemize}
\end{block}
\end{frame}
\begin{frame}

\end{frame}





\subsection{D-Latch}

\begin{frame}
Schaltung:\\




\end{frame}
\begin{frame}
\begin{block}{s}
\begin{itemize}
\item 
\end{itemize}
\end{block}
\end{frame}




\section{Zähler}
\begin{frame}
Schaltung:\\





\end{frame}
\begin{frame}
\begin{tabular}{|c|c|c|c||c|c|c|}
\hline 
$A_0$ & $A_1$ & $A_2$ & $A_3$ & $Y_0$ & $Y_1$ & $C_2$ \\ 
\hline 
0 & 0 & 0 & 0 & 0 & 0 & 0 \\ 
\hline 
1 & 0 & 0 & 0 & 1 & 0 & 0 \\ 
\hline 
1 & 1 & 0 & 0 & 1 & 1 & 0 \\ 
\hline 
1 & 1 & 1 & 0 & 0 & 0 & 1 \\ 
\hline 
1 & 1 & 1 & 1 & 0 & 1 & 1 \\ 
\hline 
0 & 1 & 1 & 1 & 1 & 0 & 1 \\ 
\hline 
0 & 0 & 1 & 1 & 1 & 1 & 0 \\ 
\hline 
0 & 0 & 0 & 1 & 0 & 1 & 0 \\ 
\hline 
0 & 1 & 0 & 0 & 0 & 1 & 0 \\ 
\hline 
0 & 0 & 1 & 0 & 1 & 0 & 0 \\ 
\hline 
0 & 1 & 1 & 0 & 1 & 1 & 0 \\ 
\hline 
0 & 1 & 0 & 1 & 0 & 0 & 1 \\ 
\hline 
1 & 0 & 1 & 0 & 0 & 1 & 0 \\ 
\hline 
1 & 1 & 0 & 1 & 1 & 0 & 1 \\ 
\hline 
1 & 0 & 0 & 1 & 1 & 1 & 0 \\ 
\hline 
1 & 0 & 1 & 1 & 0 & 0 & 1 \\ 
\hline 
\end{tabular}
\end{frame}
\end{document}
