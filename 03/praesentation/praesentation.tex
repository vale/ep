\documentclass[compress,11pt]{beamer}
%\includeonly{pendel}
\usetheme{Ilmenau}
%\usetheme{fau-4-3}
%\usecolortheme{beaver}
\beamertemplatenavigationsymbolsempty
\usepackage[ngerman]{babel}
\usepackage{marvosym}
\usepackage{multimedia}
\usepackage[utf8]{inputenc}
\usepackage{amsmath}
\usepackage{amsfonts}
\usepackage{amssymb}
\usepackage{graphicx}
\usepackage{esvect}
%\author{}
\title{EP Gruppe 8}
%\setbeamercovered{transparent}
%\setbeamertemplate{navigation symbols}{}
%\logo{}
%\institute{}
%\date{}
%\subject{}
\usepackage{verbatim}
\begin{document}




\section{Aufgabe 2}
\begin{frame}
%\includegraphics schaltung
%\end{frame}
%\begin{frame}
\begin{itemize}
\subsection{Erste Version der Schaltung}
\item Transistor hier als Schalter, da Strom nur fließt, wenn $U_{CE} \neq 0$
\item Sobald der Schalter in der ersten Schaltung geschlossen ist, liegt an Collector und Emitter eine Spannung an und der Transistor lässt durch \Rightarrow diode leuchtet
\end{itemize}
\subsection{Zweite Version der Schaltung}
\begin{itemize}
\item Jetzt liegt konstante Spannung an Collektor und Emitter \Rightarrow Transistor sperrt nicht und Diode leuchtet
\item Schaltung
\end{itemize}
\end{frame}

\end{document}
