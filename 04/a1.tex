\documentclass[compress,11pt]{beamer}
%\includeonly{pendel}
\usetheme{Ilmenau}
%\usetheme{fau-4-3}
%\usecolortheme{beaver}
\beamertemplatenavigationsymbolsempty
\usepackage[ngerman]{babel}
\usepackage{marvosym}
\usepackage{multimedia}
\usepackage[utf8]{inputenc}
\usepackage{amsmath}
\usepackage{amsfonts}
\usepackage{amssymb}
\usepackage{graphicx}
\usepackage{esvect}
%\author{}
\title{EP Gruppe 8}
%\setbeamercovered{transparent}
%\setbeamertemplate{navigation symbols}{}
%\logo{}
%\institute{}
%\date{}
%\subject{}
\usepackage{verbatim}
\begin{document}
\section{Aufgabe 1: Komparator}
\begin{frame}
Aufzubauen war folgende Schaltung:
% erste komparatorschaltung
\end{frame}

\begin{frame}
\begin{block}{Funktionsweise des Operationsverstärkers und der Schaltung}
\begin{itemize}
\item Operationsverstärker arbeitet als Differenzverstärker
\item 
\item im Idealfall gilt: Verstärkung $V \rightarrow \infty$
\item $U_{out,max}$ ist das positive/negative der Versorgungsspannung (liegt bei unseren Modell an Pin 4/7)
\end{itemize}

\end{block}
\end{frame}



\end{document}