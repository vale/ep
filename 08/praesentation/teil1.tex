\documentclass[compress,11pt]{beamer}
%\includeonly{pendel}
\usetheme{Ilmenau}
%\usetheme{fau-4-3}
%\usecolortheme{beaver}
%\beamertemplatenavigationsymbolsempty
\usepackage[ngerman]{babel}
\usepackage{marvosym}
\usepackage{multimedia}
\usepackage[utf8]{inputenc}
\usepackage{amsmath}
\usepackage{amsfonts}
\usepackage{amssymb}
\usepackage{graphicx}
\usepackage{esvect}
%\author{}
\title{EP Gruppe 8}
%\setbeamercovered{transparent}
%\setbeamertemplate{navigation symbols}{}
%\logo{}
%\institute{}
%\date{}
%\subject{}
\usepackage{verbatim}
\begin{document}
\begin{frame}

\section{A1: Digital-Analog-Wandler}

\subsection{R2R-Netzwerk}
Schaltplan:
%r2r

\begin{block}{Funktionsweise}
\begin{itemize}
\item Netzwerk aus Spannungsteilern
\item Durch Betätigen der einzelnen Schalter (0 bzw. 1) werden verschieden große Einzelströme auf den Ausgang addiert $\Rightarrow$ einer Schalterkombination ist immer ein (bis auf Messfehler) eindeutiger Ausgangsstrom zugeordnet 
\item umgekehrt kann auch einer gemessenen Strom eine Schalterkombination zugeordnet werden
\end{itemize}
\end{block}

\end{frame}
\begin{frame}{Beispiel}
Messung der Einzelströme für verschiedene Schalterkombinationen:
\begin{tabular}{|c|c|}
\hline
Zahl & I in $\mu A$  \\
\hline
0 & 0.002 \\
1 & 31.62 \\
2 & 63.8 \\
3 & 94.3 \\
4 & 125.8 \\
5 & 156.6 \\
8 & 251.8 \\
15 & 456.1 \\
\hline
\end{tabular}


\end{frame}
\subsection{}








\section{A2: Analog-Digital-Wandlung}
\begin{frame}
Schaltung:

Ausgegebene Spannung des IC kann nun am Komparator mit einer Referenzspannung verglichen werden

\end{frame}
\begin{frame}

\end{frame}
\end{document}