\documentclass[compress,11pt]{beamer}
%\includeonly{pendel}
\usetheme{Ilmenau}
%\usetheme{fau-4-3}
%\usecolortheme{beaver}
%\beamertemplatenavigationsymbolsempty
\usepackage[ngerman]{babel}
\usepackage{marvosym}
\usepackage{multimedia}
\usepackage[utf8]{inputenc}
\usepackage{amsmath}
\usepackage{amsfonts}
\usepackage{amssymb}
\usepackage{graphicx}
\usepackage{esvect}
%\author{}
\title{EP Gruppe 8}
%\setbeamercovered{transparent}
%\setbeamertemplate{navigation symbols}{}
%\logo{}
%\institute{}
%\date{}
%\subject{}
\usepackage{verbatim}
\begin{document}

\frame[c]{\titlepage}
\begin{frame}
\tableofcontents
\end{frame}

\section{Aufgabe 1}
\subsection{a)}
\begin{frame}{Innenwiderstand}
\centering
\includegraphics[width=.4\textwidth]{images/4/double-pendulum}%Schaltbild
\end{frame}

\begin{frame}
Tabellen Widerstände U|I|R
\end{frame}

\begin{frame}
%Messpunkte ohne Fit
%\includegraphics[width=\textwidth]{images/4/double-pendulum}
\end{frame}


\begin{frame}
\begin{block}{Was fällt auf?}
\begin{itemize}
\item Klicken im Messgerät an gleichen Stellen, wie Änderung des Innenwiderstands
\item 
\end{itemize}
Innenwiderstand des Messgeräts ändert sich, um größere Messbereiche abdecken zu können.
\end{block}
\end{frame}


\begin{frame}
Ein großer Strom fliest nur dann, wenn der Widerstand des Schaltkreises gering ist. Ein großer Messwiderstand hätte daher einen zu großen Anteil am Gesamtwiderstand.\\
Damit auch bei kleinen Strömen eine messbare Spannung abfällt, muss der Shunt-Widerstand entsprechend vergrößert werden.
\end{frame}


\begin{frame}
\begin{tabular}{|c|c|c|c|c|}
\hline 
IntegrationszeitMittelwert & Mittel & Standardabweichung & Minimum & Maximum \\ 
\hline 
• & -1.205800000000000e-04 & 1.612119649990745e-06 & -1.280000000000000e-04 & -1.150000000000000e-04 \\ 
\hline 
• & -0.001326910000000 & 7.561088050562645e-05 & -0.001404000000000 & -1.850000000000000e-04 \\ 
\hline 
• & -0.007228985336667 & 0.011961495895749 & -0.027426000000000 & 0.014268000000000 \\ 
\hline 
• &  -0.013617053333333 & 0.012538989757190 &  -0.036758000000000 & 0.009862000000000 \\ 
\hline 
• & -0.017347804271100 & 0.012955625955849 & -0.041245000000000 &  0.006758000000000 \\ 
\hline 
• & -0.016173373974533 & 0.012837624964948 & -0.040475000000000 & 0.008057000000000 \\ 
\hline 
• & -9.605146482333332e-04 & 0.002075512735832 & -0.004252000000000 & 0.002389000000000 \\ 
\hline 
\end{tabular} 
\end{frame}

\begin{frame}
\includegraphics[width=\textwidth]{images/4/tracker}
\end{frame}

\subsection{Auswertung}
\begin{frame}
\begin{columns}
\column{0.5\textwidth}
\includegraphics[width=\textwidth]{images/4/pendel-3}\\
\includegraphics[width=\textwidth]{images/4/pendel-7}
\column{0.5\textwidth}
\includegraphics[width=\textwidth]{images/4/pendel-8}\\
\includegraphics[width=\textwidth]{images/4/pendel-9}
\end{columns}
\end{frame}

\subsection*{Masse $m_1$ -- Messung und Simulation}
\begin{frame}
\includegraphics[width=\textwidth]{images/4/phi1uebert}
\end{frame}

\subsection*{Masse $m_1$ -- Phasenraumdiagramm}
\begin{frame}
\includegraphics[width=\textwidth]{images/4/ph1_ueberphi1}
\end{frame}

\subsection*{Masse $m_2$ -- Messung und Simulation}
\begin{frame}
\includegraphics[width=\textwidth]{images/4/phi2uebert}
\end{frame}

\subsection*{Masse $m_2$ -- Phasenraumdiagramm}
\begin{frame}
\includegraphics[width=\textwidth]{images/4/phi2_ueberphi2}
\end{frame}

\begin{frame}
\movie[showcontrols=false, autostart]{\includegraphics[width=\textwidth]{movies/show.jpg}}{movies/show.mov}
\end{frame}
\end{document}
