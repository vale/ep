\documentclass[compress,11pt]{beamer}
%\includeonly{pendel}
\usetheme{Ilmenau}
%\usetheme{fau-4-3}
%\usecolortheme{beaver}
%\beamertemplatenavigationsymbolsempty
\usepackage[ngerman]{babel}
\usepackage{marvosym}
\usepackage{multimedia}
\usepackage[utf8]{inputenc}
\usepackage{amsmath}
\usepackage{amsfonts}
\usepackage{amssymb}
\usepackage{graphicx}
\usepackage{esvect}
%\author{}
\title{EP Gruppe 8}
%\setbeamercovered{transparent}
%\setbeamertemplate{navigation symbols}{}
%\logo{}
%\institute{}
%\date{}
%\subject{}
\usepackage{verbatim}
\begin{document}

\frame[c]{\titlepage}
\begin{frame}
\tableofcontents
\end{frame}

\section{Aufgabe 1}
\subsection{a)}
\begin{frame}{Innenwiderstand}
\centering
\includegraphics[width=.4\textwidth]{images/4/double-pendulum}%Schaltbild
\end{frame}

\begin{frame}
Tabellen Widerstände U|I|R
\end{frame}

\begin{frame}
%Messpunkte ohne Fit
%\includegraphics[width=\textwidth]{images/4/double-pendulum}
\end{frame}


\begin{frame}
\begin{block}{Was fällt auf?}
\begin{itemize}
\item Klicken im Messgerät an gleichen Stellen, wie Änderung des Innenwiderstands
\item 
\end{itemize}
Innenwiderstand des Messgeräts ändert sich, um größere Messbereiche abdecken zu können.
\end{block}
\end{frame}


\begin{frame}
Ein großer Strom fliest nur dann, wenn der Widerstand des Schaltkreises gering ist. Ein großer Messwiderstand hätte daher einen zu großen Anteil am Gesamtwiderstand.\\
Damit auch bei kleinen Strömen eine messbare Spannung abfällt, muss der Shunt-Widerstand entsprechend vergrößert werden.
\end{frame}


\begin{frame}
\begin{tabular}{|c|c|c|c|c|}
\hline 
IntegrationszeitMittelwert & Mittel & Standardabweichung & Minimum & Maximum \\ 
\hline 
• & -1.205800000000000e-04 & 1.612119649990745e-06 & -1.280000000000000e-04 & -1.150000000000000e-04 \\ 
\hline 
• & -0.001326910000000 & 7.561088050562645e-05 & -0.001404000000000 & -1.850000000000000e-04 \\ 
\hline 
• & -0.007228985336667 & 0.011961495895749 & -0.027426000000000 & 0.014268000000000 \\ 
\hline 
• &  -0.013617053333333 & 0.012538989757190 &  -0.036758000000000 & 0.009862000000000 \\ 
\hline 
• & -0.017347804271100 & 0.012955625955849 & -0.041245000000000 &  0.006758000000000 \\ 
\hline 
• & -0.016173373974533 & 0.012837624964948 & -0.040475000000000 & 0.008057000000000 \\ 
\hline 
• & -9.605146482333332e-04 & 0.002075512735832 & -0.004252000000000 & 0.002389000000000 \\ 
\hline 
\end{tabular} 
\end{frame}

\begin{frame}
%\includegraphics[width=\textwidth]{images/4/tracker}
\end{frame}

\section{Aufgabe 2}
\begin{frame}
\subsection{Schaltung 1}
\subsubsection{Widerstände}

Alle Werte sind in $\Omega$ angegeben. Für die berechneten Widerstände wurden die an den jeweiligen Stromstärken verwendet.\\
\begin{tabular}{|c|c|c|c|}
\hline 
- & Erwarteter Widerstand & Berechneter " & Gemessener " \\ 
\hline 
$R_{ges}$  & 97916.7 & 100000 & 96586.7 \\ 
\hline 
$R_1$ & 4700000 & 4652790.7 & 4640000 \\ 
\hline
$R_2$ & 100000 & 100035 & 98640 \\
\hline
\end{tabular}

\subsection{Stromstärken an den Widerständen}
Erhält man über den folgenden Ansatz:
\begin{equation}
U = const. = I_1 \cdot R_1 = I_2 \cdot R_2
\end{equation}
\begin{equation}
\Rightarrow \frac{I_1}{I_2} = \frac{R_2}{R_1} = \frac{1 k\Omega}{4.7 M\Omega} = \frac{1}{47}
\end{equation}
\begin{equation}
I_{ges} = I_1 + I_2 = \frac{I_2}{47} + I_2
\end{equation}
\begin{equation}
\Rightarrow I_2 = \frac{47}{48} \cdot I_{ges} = \frac{47}{48} \cdot 0.2 mA = 1.9583 e-4
\end{equation}
\begin{equation}
I_1 = I_{ges} - I_2 = 4.167 \mu A
\end{equation}
Werte stimmen bis auf kleine Ungenauigkeiten mit den gemessenen Werten überein.
\subsection{Leistung an den Widerständen}
\begin{equation}
P_1 = I_1^2 \cdot R_1 = 8.16 e-5 W
\end{equation}
\begin{equation}
P_2 = I_2^2 \cdot R_2 = 0.004 W
\end{equation}
\end{frame}

\begin{frame}
\subsection{Schaltung 2}
$R_{ges} = 500 \Omega$ \\ Mit $U_{ges,genmessen} = 0.843 V$ und $I_{ges,gemessen} = 1.7 mA$ ergibt sich: 
\begin{equation}
R_{ges} = \frac{U_{ges}}{I_{ges}} = 495.9 \Omega
\end{equation}
\begin{equation}
R_{1,gem} = 989 \Omega 
\end{equation}
\begin{equation}
R_{2,gem} = 992 \Omega
\end{equation}
\begin{equation}
\Rightarrow R_{ges,gem} = \frac{R_{1,gem} \cdot R_{2,gem}}{R_{1,gem} + R_{2,gem}} = 495.25 \Omega
\end{equation}
\end{frame}
\begin{frame}
\subsection{Schaltung 3}
$R_{ges} = 9.4 M \Omega$, $I_{ges,gem} = 2.14 \mu A$, $U_{ges,gem} = 2.006 V$
Es ergibt sich:
\begin{equation}
R_1 = \frac{U_1}{I} = 3802336.5 \Omega
\end{equation}
\begin{equation}
R_2 = 3779906.5 \Omega
\end{equation}
Gemessene Widerstände sind zu klein, da der tatsächliche Widerstand des Stromkreises im Verhältnis zum Innenwiderstand des DMM zu groß ist, d.h. es fällt zu viel Spannung am DMM ab. Mit dem Modus "HI-Z" wird der Innenwiderstand auf 10 G$\Omega$ erhöht und es ergibt sich:
\begin{equation}
R_1^{(HI-Z)} = \frac{U_1^{(HI-Z)}}{I} = \frac{10.033 V}{I}= 4688317.8 \Omega
\end{equation}
\begin{equation}
R_2^{(HI-Z)} = \frac{9.937 V}{I}=4660280 \Omega
\end{equation}
\end{frame}

\subsection*{Masse $m_1$ -- Phasenraumdiagramm}
\begin{frame}
%\includegraphics[width=\textwidth]{images/4/ph1_ueberphi1}
\end{frame}

\subsection*{Masse $m_2$ -- Messung und Simulation}
\begin{frame}
%\includegraphics[width=\textwidth]{images/4/phi2uebert}
\end{frame}

\subsection*{Masse $m_2$ -- Phasenraumdiagramm}
\begin{frame}
%\includegraphics[width=\textwidth]{images/4/phi2_ueberphi2}
\end{frame}

\begin{frame}
%\movie[showcontrols=false, autostart]{\includegraphics[width=\textwidth]{movies/show.jpg}}{movies/show.mov}
\end{frame}
\end{document}
